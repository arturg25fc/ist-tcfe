\section{Simulation Analysis}
\label{sec:simulation}
\captionsetup[table]{skip=10pt}

\subsection{Operating Point Analysis}

\vspace{5mm}
\par Table \ref{tab_op} shows the simulated operating point results (obtained from Ngspice) for the circuit under analysis, considering that current directions $I_1$, $I_2$, $I_3$ and $I_4$ (and, accordingly, the positive and negative terminals of the components) were defined based on the same model as before.


\renewcommand{\arraystretch}{1.5}
\begin{table}[h]
  \centering
  \begin{tabular}{|c|c|}
    \hline    
    \textbf{Variable name} & \textbf{Value [mA or V]} \\ \hline
    @gb[i] & -2.00589e-01\\ \hline
@id[current] & 1.017967e+00\\ \hline
@r1[i] & 1.915709e-01\\ \hline
@r2[i] & -2.00589e-01\\ \hline
@r3[i] & -9.01783e-03\\ \hline
@r4[i] & 1.164729e+00\\ \hline
@r5[i] & -1.21856e+00\\ \hline
@r6[i] & 9.731579e-01\\ \hline
@r7[i] & 9.731579e-01\\ \hline
v(1) & 8.030092e+00\\ \hline
v(2) & 7.833907e+00\\ \hline
v(3) & 7.417135e+00\\ \hline
v(4) & 2.989101e+00\\ \hline
v(5) & 7.861734e+00\\ \hline
v(6) & 1.163284e+01\\ \hline
v(7) & 9.951293e-01\\ \hline
v(8) & 2.989101e+00\\ \hline

  \end{tabular}
  \caption{Operating point analysis - a variable preceded by @ is of type \textit{current} and expressed in miliAmpere; other variables are of type \textit {voltage} and expressed in Volt. $gb$ refers to the controlled current source $I_b$ and the rest is defined as before.}
  \label{tab_op}
\end{table}
\vspace{3mm}
\par Compared to the theoretical analysis results, one notices a few differences.
\vspace{3mm}
\par Focusing on the obtained nodal voltage levels, an almost exact resemblance is shown, except for the fact that Octave's used result precision is of 6 decimal places, whereas Ngspice always presents 7 significant digits (hence the small differences). It's worth noting that node 8 (introduced exclusively on the simulation analysis) was created so a zero valued voltage source (working as an ammeter) could be placed in series with resistor 6, allowing the algorithm to measure the current between nodes 4 and 7.
\vspace{3mm}
\par The same issue takes place with the current values, altough we can only directly compare four of them: $I_1$ with $@r2[i]$, $I_2$ with $@r1[i]$, $I_3$ with both $@r6[i]$ and $@r7[i]$ as well as $I_4$ with the symmetric of $I_d$, as stated on section \ref{sec:analysis} (even if it was rather an input value and not quite an output one). For further comparisons, proceed as indicated in sections \ref{sec:2.1} and \ref{sec:2.2}.
\vspace{3mm}
\par Other than that, we could say there's a perfect match on the results obtained from both analysis methods.
\vspace{5mm}





