
\section{Theoretical Analysis}
\label{sec:analysis}
\vspace{3mm}
\par In this section, the previously shown circuit is theoretically analised.
\vspace{3mm}
\par First of all, it is important to refer that the transformer had to be replaced by a pair of dependent sources, so that the value of the input voltage could be modified. The following figure shows the resulting circuits.

\begin{figure}[h] \centering
\includegraphics[scale=0.5]{t3ds.pdf}
\caption{AC/DC Converter considered with the transformer equivalent dependent sources.}
\label{fig:t3ds}
\end{figure}

\vspace{100mm}

\begin{table}[h]
\centering
    \begin{tabular}{c|c}
      Quantity name  & Value\\ \hline
      C & 600 $\mu$ F\\
      R & 600 $k\Omega$ \\
      Cost & 1201.6 MU \\
    \end{tabular}
    \caption{Values of circuit quantities.}
 \label{tab_stefan}
\end{table}

After, a rectifier bridge, built with four diodes, allows our signal to be full wave rectified. Its output is nothing less than the absolute value of $v_S$, being the first step towards a more uniform signal. \par An \textbf{envelop detector} containing the rectifier and a capacitor covers the signal and produces an output that oscillates between charging and discharging periods to a resistor. Note that this discharge is allowed at $t=t_{OFF}$, as the diodes go OFF.

\begin{equation}
  t_{OFF}=\frac{1}{\omega}*arctan(\frac{1}{\omega RC})
\end{equation}

allowing

\begin{equation}
  v_C(t)=Acos(\omega t_{OFF})e^{-\frac{t-t_{OFF}}{RC}}
\end{equation}

after which, the recharge is possible due to the reactivation of the diodes.
\par

In order to reduce the remaining ripple, a \textbf{voltage regulator} circuit built with a series of 16 diodes increases the time constant. By means of incremental analysis (application to the 16 series diodes as resistors $r_d$), the incremental output $v_O$ can be calculated with

\begin{equation}
  v_O=\frac{16r_d}{16r_d+R}v_C
\end{equation}

\begin{equation}
  r_d=\frac{\eta V_T}{I_Se^{\frac{V_d}{\eta V_T}}}
\end{equation}

with $r_d$ being the incremental resistance, $\eta=1$ a material constant and $V_T$ the termic voltage.

\par

To calculate the ripple, it is used

\begin{equation}
 v_{ripple}=max(V_{DC})-min(V_{DC})
\end{equation}

\vspace{100mm}
