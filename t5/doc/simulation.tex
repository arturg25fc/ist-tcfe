\section{Simulation Analysis and result comparison}
\label{sec:simulation}
\captionsetup[table]{skip=10pt}

\vspace{5mm}


\renewcommand{\arraystretch}{1.5}

The following table presents the output voltage gain in the passband, the central frequency and the input and output impedances at this frequency.


\begin{table}[htbp]
\begin{minipage}{0.5\linewidth}
\centering
\begin{tabular}{|c|c|}
\hline    
\textbf{Quantity name} & \textbf{Value [V, MU, Hz or other]} \\ \hline
gaindb & 3.876491e+01\\ \hline
gaindeviation & 1.379498e+01\\ \hline
cutofflow & 4.050703e+02\\ \hline
cutoffhigh & 2.426023e+03\\ \hline
centralfreq & 9.913172e+02\\ \hline
centralfreqdeviation & 8.682763e+00\\ \hline
cost & 1.345695e+04\\ \hline
merit & 3.305983e-06\\ \hline

zin & 9.999843e+02,-7.23564e+02\\ \hline
zinabs & 1.234307e+03\\ \hline

zout & 6.810520e+02,-4.66790e+02\\ \hline
zoutabs & 8.256660e+02\\ \hline

\end{tabular}
\end{minipage}
\hspace{15mm}
\begin {minipage}{0.5\linewidth}
\centering
\begin{tabular}{|c|c|}
\hline    
\textbf{Quantity name} & \textbf{Value [V, MU, Hz or other]} \\ \hline
$gain_{db}$ & 3.882159e+01 \\ \hline
$gain_{deviation}$ & 1.268688e+01 \\ \hline
$cutoff_{low}$ & 4.088948e+02\\ \hline
$cutoff_{high}$ & 2.572826e+03 \\\hline
$central_{freq}$ & 1.025678e+03 \\\hline
$central_{freq_{deviation}}$ & 2.567793e+01 \\\hline
zin & 1.000e+03-7.234e+02 i \\\hline
abs(zi) & 1.234242e+03\\\hline
zo & 6.767e+02-4.677e+02 i \\\hline
abs(zo) & 8.226375e+02 \\\hline

\end{tabular}
\end{minipage}
\caption{On the left, the values obtained via simulation in Ngspice. On the right, the ones obtained from Octave.}
\label{tab_1}
\end{table}


\begin{minipage}[b]{0.5\linewidth}
\centering
\includegraphics[width=0.95\linewidth]{phase.eps}
\captionsetup{type=figure}
\caption{Plot for the gain (dB), obtained with Octave.}
\label{gain_octave}
\end{minipage}
\begin{minipage}[b]{0.5\linewidth}
\centering
\includegraphics[width=0.95\linewidth]{vdbf.pdf}
\captionsetup{type=figure}
\caption{Plot for the gain (dB), obtained with Ngspice.}
\label{gain_ngspice}
\end{minipage} 
\newpage
\begin{minipage}[b]{0.5\linewidth}
\centering
\includegraphics[width=0.95\linewidth]{gain.eps}
\captionsetup{type=figure}
\caption{Plot for the phase (degrees), obtained with Octave.}
\label{vph_octave}
\end{minipage}
\begin{minipage}[b]{0.5\linewidth}
\centering
\includegraphics[width=0.95\linewidth]{vphvof.pdf}
\captionsetup{type=figure}
\caption{Plot for the phase (degrees), obtained with Ngspice.}
\label{ph_ngspice}
\end{minipage} 


\vspace{3mm}
\par Comparing both the simulation and theoretical results, one notices a few differences. In the theoretial analysis, it was assumed an ideal model for the OP-AMP, with perfect impedances (as explained in theoretical analysis). In adition, non-linear components were linearly analysed. On the other hand, Ngspice makes use of a much more sofisticated model with a meriad of parameters.
\vspace{3mm}
\par Related to the phase plot: Octave's shows the evidence of 2 poles (2 capacitors), with a drop of 180 degreees (from 90 to -90 degrees). On the other side, Ngspice's one shows the evidence of 4 poles (4 capacitors considered on the simulation model). Besides that, due to representation range issues, there is a discontinuity in Ngspice's plot although the real variation is continuous.    
\vspace{5mm}